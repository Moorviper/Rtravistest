\section{Grundlagen}
\setlength{\columnseprule}{1px}
\subsubsection{Brüche}
Zähler x Zähler\\
Nenner x  Nenner \\

Brüche werden dividiert in dem mal den Kehrwert multipliziert.\\

\subsection{Potenzieren und Exponenten}

\begin{multicols}{3}
  {\color{fulda_green}Positiver} Exponent\\
  \newline
  \columnbreak
  $a^n = \underbrace{a \cdot a \cdot a ...}_{n Faktoren}$ \\
  {\color{red}Negativer} Exponent\\
  \newline
  \columnbreak
  $n^{-n} = \frac{1}{a^n}$ \\
  {\color{monoorange}Rationaler} Exponent\\
  \newline
  $a^{\frac{m}{n}} = \sqrt[n]{m} = (\sqrt[n]{a})^m$ \\
\end{multicols}

\subsection{Potenzgesetze}
Gleiche Basis
% \setlength{\columnsep}{2px}
%\begin{multicols}{Spaltenzahl}[''titel''][''Abstand'']

\begin{multicols}{2}
    \centering{$a^p \cdot a^q = a^{p + q}$}\\
    \columnbreak
  $a^p : a^q = \frac{a^p}{a^q} = a^{p-q} $
\end{multicols}

Gleiche Exponenten
\begin{multicols}{2}
    \centering{$a^p \cdot b^p = (a \cdot b)^p$}\\
    \columnbreak
  $a^p : b^p = \frac{a^p}{b^p} = (\frac{a}{b})^p $
\end{multicols}

Potenzieren von Potenzen
\begin{align*}
  (a^p)^q &= a^{p \cdot q}
\end{align*}


\begin{multicols}{2}
    \centering{$a^{\frac{1}{n}} = \sqrt[n]{a} $}\\
    \columnbreak
  $a^{-\frac{1}{n}} = \frac{1}{\sqrt[n]{a}} $
\end{multicols}
