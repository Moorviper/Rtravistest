% arara: lualatex: { shell: yes, action: nonstopmode, synctex: yes}
% arara: lualatex: { shell: yes, action: nonstopmode, synctex: yes}
% asarara: lualatex: { shell: yes, action: nonstopmode, synctex: yes,  options: "-output-directory=_build"}
% asarara: lualatex: { shell: yes, action: nonstopmode, synctex: yes,  options: "-output-directory=_build"}
\documentclass{article}
\usepackage[ngerman]{babel}
\usepackage[no-math]{fontspec}

\usepackage{mwe}
\usepackage{luacode}
\usepackage{shellesc}
% \documentclassw[11pt, a4paper,ngerman]{article}
\usepackage{basicff}

\usetikzlibrary{patterns} % preamble
\tcbuselibrary{skins} % preamble

\usepackage{tikz}
% \usepackage{PTSansNarrow}
\usetikzlibrary{matrix}

\usepackage{pgfplots}
\pgfplotsset{
 compat=newest
  }
\tcbset{colframe=red!75!black}
\newenvironment{proggen}{\begin{center}}{\end{center}}

\usepackage{array}
% \setmainfont[Path=/Applications/Microsoft Word.app/Contents/Resources/Fonts/]{Calibri.ttf}
% \setsansfont[Path=/Applications/Microsoft Word.app/Contents/Resources/Fonts/]{Calibri.ttf}
% \setmonofont[Path=/Applications/Microsoft Word.app/Contents/Resources/Fonts/]{Calibri.ttf}


\setmainfont{UbuntuL.ttf}
\setsansfont{UbuntuR.ttf}
\setmonofont{UbuntuMonoR.ttf}

\newcolumntype{C}[1]{>{\centering\arraybackslash}m{#1}}

\oddsidemargin-10mm
\title{
\color{white}
 $\bullet$ \\ $\bullet$ \\ $\bullet$ \\
 \color{black}
 % \color{white}
 % $\bullet$ \\
 \color{black}
 Mathematik 1\\
Zusammenfassung \\
Schöppner \\
\color{white}
$\bullet$ \\
\color{black}
 \begin{center}
	% \includegraphics[scale=0.5]{./pictures/CaptainWarschburger.jpg}
\end{center}
 % \includegraphics{./pictures/wohnzimmer.png}
}
% \includegraphics{./pictures/asrock.png} Q1900M \\ \color{white} $\bullet$ \\ $\bullet$ \\ $\bullet$ \\ \color{black} \\ \apple \\ 10.10.3

\author{Daniel Krah}
% \date{1.6.2015}

\begin{document}
% \AddToShipoutPicture{\BackgroundPic}
\maketitle%
% \newpage%
%  \tableofcontents%
% \newpage
%==================================================================================
% \begin{center}
% \textbf{Vorwort}
% \end{center}



%
% \input{hardware.tex}







%  To Do
% \input{todo.tex}
\newpage
% \section{Überblick}


% \begin{displaymath}
%   E = \frac{m_{0} c^{2}}{\sqrt{1-v^{2}/c^{2}}}
% \end{displaymath}

% \begin{luacode}
%   for x=1,600 do
%     tex.print(x+2)
%   end
% \end{luacode}
\section{Grundlagen}
\setlength{\columnseprule}{1px}
\subsubsection{Brüche}
Zähler x Zähler\\
Nenner x  Nenner \\

\monocodebox{html}{HTML}{./code/html.html}{false}{1}{9999999}
\monocodebox{css}{CSS}{./code/css.css}{false}{1}{9999999}
\monocodebox{js}{JavaScript}{./code/javascript.js}{false}{1}{4}


Test

% 
\section{Aufleiten}

\begin{align*}
f(x) &= 3x -4 \\
     &= 3x^1 -4x^0 \\
F(x) &= \frac{3}{2}x^2 - \frac{4}{1} x^1 +k
\end{align*}

\begin{align*}
  f(x) &= 5 \sqrt{x} = 5x^{0,5} \\
  F(x) &= \frac{1}{1,5}* 5*x^{1,5}  = \frac{2}{3} * 5 * x^{1,5} \\
  F(x) &= \frac{10}{3} x^{1,5} + k
\end{align*}

\begin{align*}
  f(x) &=  \frac{2}{3x^2} = \frac{2}{3}x^{-2} \\
  F(x) &= \frac{2}{3} * \frac{1}{-1} x^{-2+1} \\
  F(x) &= - \frac{2}{3} x^{-1} + k \\
  \\
  f(x) &= \frac{1}{x} \\
  F(x) &= ln |x| +k
\end{align*}


sin -cos -sin  cos sin .... \\

\subsection{Kettenregel}
Ableiten: mal die innere Ableitung\\
Aufleiten: durch die innere Ableitung\\


\begin{align*}
  f(x) &=  (2x + 3)^2\\
  F(x) &= \frac{1}{3} (2x +3)^3 * \frac{1}{2} + k \\
  F(x) &= \frac{1}{6} (2x +3)^3 +k
\end{align*}
Innere ableitung von 2x = 2 daher $* \frac{1}{2}$
Innere Funktion muss linear sein \\

\begin{align*}
  f(x) &= 2 cos(3x)\\
  F(x) &= 2 sin(3x) * \frac{1}{3} +k \\
  F(x) &= \frac{2}{3} sin(3x) + k
\end{align*}

\subsection{Ableiten}
1. Ableitung = Steigung \\
Extrempunkt Ableitung = 0 \\
\begin{align*}
  f(x) &= x^3 - 3x^2 \\
  f`(x) &= 3x^2 - 6x = 0 \\
  f`(x) &= x (3x - 6) = 0 \\
  x1 &= 0 \\
  x2 &= (3x - 6) = 2\\
\end{align*}

\subsection{Hoch-/Tiefpunkt \Rightarrow 2. Ableitung}
\begin{align*}
  f``(x) > 0 &\rightarrow \text{Tiefpunkt} \\
  f``(x) < 0 &\rightarrow \text{Hochpunkt}
\end{align*}

\begin{align*}
  f`(x) &= 3x^2 - 6x\\
  f``(x) &= 6x - 6 \\
  x1 &= 0 \rightarrow  f``(0) = -6 \ \text{<0 Hochpunkt} \\
  x2 &= 2 \rightarrow  f``(2) = 6 \ \text{>0 Tiefpunkt}
\end{align*}

\subsection{Wendepunkte 2. Ableitung = 0 , 3. Ableitung $\neq{}$ 0}
Links nach rechts $f```(x) < 0$ \\
Rechts nach links $f```(x) > 0$ \\
Sattelpunkt
\begin{align*}
  f`(x) &= 0 \\
  f``(x) &= 0 \\
  f```(x) & \neq{} 0 \\
\end{align*}

Ungrade Exponenten größer 1 immer 3 Schnittpunkte.

\section{Integrale}
\begin{align*}
\int_0^3 f(x) &dx \\
\int_1^3 \frac{1}{2}x^2 &dx = F(3) - F(1) = [F(x)]_1^3 \\
F(x) &= \frac{1}{6} x^3 +k \\
F(3) &= \frac{1}{6} 3^3 +k  = \frac{27}{6} \\
F(1) &= \frac{1}{6} 1^3 +k  = \frac{1}{6} \\
&= \frac{26}{6} = \frac{13}{3}
\end{align*}

Bei 2 Funktionen nie über schnittpunkt \\

\section{Exponentialfunktion und Logarithmus}

\begin{align*}
\lim\limits_{n \to \infty}(1 + \frac{1}{n})^2 = 2,71828 \\
e &= 2,7 \\
f(x) &= 3^x \\
f(x) &= e^x  \\
f(x) &= k^x \\
f(0) &= k^0 = 1 \ \text{immer} \\
\\
f(x) &= x^2 \\
f(-1) &= 2^{-1} = \frac{1}{2} = 0,5 \\
f(-5) &= \frac{1}{32} = 0,0312
\end{align*}

\subsection{Regeln}
\begin{align*}
  x^2 + x^2 &= x^4 \\
  3^a * 3^b &= 3^{a+b} \\
  \frac{3^a}{3^b} & = 3^{a-b}\\
  3^{a^b} &= 3^{a*b} \\
  3^a + 3^b &= \text{Geht nicht}
\end{align*}

\subsection{Logarithmus}
\begin{align*}
  log_10 (1000) &= 3 \\
  10^3 &= 1000 \\
  log_2 (4) &= 2 \\
  2^2 &= 4 \\
  log_10 (10^x) &= x \\
  ln(e^x) &= x \\
  e^{ln(x)} &= x \\
  \\
  ln(1) &= 0 \\
  ln(e) &= 1 \\
  \\
  f(x) &= e^x \\
  f(x) &= ln(x)%\text{Umkehrfunktion}
\end{align*}

\section{Lösen von quadratischen Gleichungen}

Allgemeine Form:
\begin{align*}
  ax^2 + bx + c &= 0\\
\end{align*}
Normalform:
\begin{align*}
  x^2 + px + q &= 0\\
\end{align*}
\subsection{Lösen von quadratischen Gleichungen}
\subsubsection{PQ Formel}
\begin{align*}
  x_{1,2} &= -\frac{p}{2} +/- \sqrt{(\frac{p}{2})^2 - q}
\end{align*}
\subsubsection{Mitternachtsformel}
\begin{align*}
  ax^2 + bx + c &= 0\\
  x_{1,2} &= \frac{-b \pm \sqrt{b^2 - 4ac}}
                  {2a}
\end{align*}


\section{Vector}

Punkt A=(0/0/0)\\
Punkt B=(4/5/5)\\

$x_1 ,x_2 ,x_3$ yxz\\
vector\\
\begin{align*}
  \frac{\rightarrow}{b} = \left(
   \begin{array}{c}
     4 \\
     5 \\
     5
   \end{array}
\right)
= \frac{\rightarrow}{AB}
\end{align*}
Länge vecotr = betrag\\
\begin{align*}
  \frac{\rightarrow}{b} = \left(
   \begin{array}{c}
     1 \\
     2 \\
     3
   \end{array}
\right)
\frac{\rightarrow}{|b|} = \sqrt{1^2 +2^2 + 3^3}
\end{align*}

wenn:\\
\begin{align*}
  \frac{\rightarrow}{a} * \frac{\rightarrow}{b} = 0 \ \text{dann senkrecht}
\end{align*}

Winkel zwischen 2 vectoren

\begin{align*}
  cos(alpha) = \frac{\frac{\rightarrow}{a} * \frac{\rightarrow}{b} }{|\frac{\rightarrow}{a}| * |\frac{\rightarrow}{b}|}
\end{align*}

Skalarprodukt von senkrechten Vectoren = 0 \\

\begin{align*}
  \frac{\rightarrow}{a} = \left(
   \begin{array}{c}
     1 \\
     -3 \\
     2
   \end{array}
\right)
\end{align*}
\begin{align*}
  \frac{\rightarrow}{b} = \left(
   \begin{array}{c}
     -2 \\
     6 \\
     -4
   \end{array}
\right)
\end{align*}
\begin{align*}
  |\frac{\rightarrow}{a}| = \sqrt{1^2 + (-3)^2 +2^2} = \sqrt{14} \\
  |\frac{\rightarrow}{b}| = \sqrt{(-2)^2 + 6^2 +(-4)^2} = \sqrt{56}
\end{align*}
\begin{align*}
  \frac{\rightarrow}{a} * \frac{\rightarrow}{b} &= \left(
   \begin{array}{c}
     1 \\
     -3 \\
     2
   \end{array}
\right) * \left(
 \begin{array}{c}
   -2 \\
   6 \\
   -4
 \end{array}
\right) \\
&= 1*(-2)+(-3)*6+2*(-4) = -28
\end{align*}
3. Einsetzen

\begin{align*}
  cos(alpha) &= \frac{-28}{\sqrt{14} * \sqrt{56}}\\
  &= -1 \\
  alpha &= arccos(-1) = 180^0
\end{align*}

\begin{center}
  \includegraphics[width=0.8\textwidth]{./pictures/1.png}
\end{center}
\begin{center}
  \includegraphics[width=0.8\textwidth]{./pictures/2.png}
\end{center}
\begin{center}
  \includegraphics[width=0.8\textwidth]{./pictures/3.png}
\end{center}
\begin{center}
  \includegraphics[width=0.8\textwidth]{./pictures/4.png}
\end{center}
\begin{center}
  \includegraphics[width=0.8\textwidth]{./pictures/1.jpg}
\end{center}
\begin{center}
  \includegraphics[width=0.8\textwidth]{./pictures/2.jpg}
\end{center}
\begin{center}
  \includegraphics[width=0.8\textwidth]{./pictures/3.jpg}
\end{center}
\begin{center}
  \includegraphics[width=0.8\textwidth]{./pictures/4.jpg}
\end{center}
\begin{center}
  \includegraphics[width=0.8\textwidth]{./pictures/5.jpg}
\end{center}
\begin{center}
  \includegraphics[width=0.8\textwidth]{./pictures/6.jpg}
\end{center}
\begin{center}
  \includegraphics[width=0.8\textwidth]{./pictures/7.jpg}
\end{center}
\begin{center}
  \includegraphics[width=0.8\textwidth]{./pictures/8.jpg}
\end{center}
\begin{center}
  \includegraphics[width=0.8\textwidth]{./pictures/9.jpg}
\end{center}
\begin{center}
  \includegraphics[width=0.8\textwidth]{./pictures/10.jpg}
\end{center}
\begin{center}
  \includegraphics[width=0.8\textwidth]{./pictures/11.jpg}
\end{center}

% \input{kap1.tex}
uhjh


% \input{dummy.tex}
% \input{dummy.tex}
% \input{dummy.tex}
% \input{dummy.tex}

% \input{prog1.tex}



%  die beiden unteren beiden includen

% \input{kap1_vorl.tex}
% \input{kap2_vorl.tex}
% \input{kap3_vorl.tex}
% \input{kap4_vorl.tex}
% \input{kap5_vorl.tex}
% \input{kap6_vorl.tex}
% kapitel 7 im Buch

%   Kapitel 2

% \input{2ndKap7.tex}




%==============================================

% % programmieren 2

% \input{kap8.tex}
% \input{kap9.tex}
% \input{kap10.tex}
% \input{kap11.tex}
% \input{kap12.tex}
% \input{last.tex}
% \input{einfuerung.tex}




%=========================================================================================================================
%
%-------------END
\end{document}



% \begin{Verbatim}[frame=lines,
%        framerule=0.2mm,framesep=3mm,
%        rulecolor=\color{monoorange},
%        fillcolor=\color{monogreen},
%        label=Kapitel 1,labelposition=topline]
%   Name    :   Cubieboard 2
%   Size    :   10 cm x 6 cm
%   CPU     :   Allwinner A20 SoC (2 ARM-Cortex A7-Cores with 1 GHz)
%   GPU     :   Mali-400MP2 (OpenGL ES 2.0/1.1)
%   VPU     :   CedarX (max 2160p (Ultra HD))
%   RAM     :   512MB (Test) / 1GB (Produktion) DDR3
%   CONN    :   2x USB Host, 1x USB On-the-go, 1x CIR, 1x SATA
%   VID-OUT :   HDMI @ 1080p
%   AUD-OUT :   S/PDIF, Headphone, HDMI-Audio
%   AUD-IN  :   Mikrophone, Line-In
%   Storage :   4 GB NAND-Flash, 1x MicroSD
%   Network :   10/100-Ethernet
%   DB-Con. :   96 Pin incl I²C, SPI, LVDS
% \end{Verbatim}








% z23dsdsdsdsd
% \color{red}
% \shiftkey
% \capslockkey
% \ejectkey
% \pencilkey
% \returnkey
% \revreturnkey
% \cubie
% \cubiebig
% \archlinux \\

% \subsection{Notes}
% \infobox{Notes}{
% $\ $\\
%  \centering
% vdpau-sunxi is still in development. \\
% $\ $\\
% $\ $\\
% }{$\ $}



%hjhjj
 % \input{aufgabe29.tex}







% sss



% \newpage
% 18.2.

% klausur

% 10:15 - 11:45

% ssdddsd

% 26. Das Projekt Technischer Kundendienst liegt in der Version 3 im Verzeichnis O:\Paul\ Programmierung1\Projekte \Kapitel5, bzw. in S2T. \\

% a. Kopieren Sie das Projekt in Ihr Verzeichnis. Machen Sie sich mit dem Projekt vertraut und spielen Sie die Anwendung durch. \\
% b. Schauen Sie sich die Klasse Beantworter an. Erklären Sie Ihrem Nachbarn den Ablauf der Methode generiereAntwort() der Klasse Beantworter und den genauen Zweck der Sammlung AntwortMap. \\

% c. Stellen Sie sich vor, eine HashMap hätte für die vorliegende Anwendung nicht zur Verfügung gestanden. Hätte man die Aufgabe auch mit einer ArrayList lösen können? \\
% Überlegen Sie sich mit Ihrem Nachbarn, wie eine solche alternative Lösung im Prinzip aussehen würde. \\


% 27. Der Umgang mit Zeichenketten vom Typ String soll an Hand einiger typischer Aufga- ben durchgespielt werden. Nehmen Sie für die folgenden Aufgaben die Dokumentation der Klasse String zur Hand. \\

% a. Kopieren Sie das Projekt Zeichenketten aus dem O:\Paul\Programmierung1\Projekte\ Kapitel05.
% b. Die Klasse StringBearbeitung enthält zahlreiche Methodenentwürfe, deren Implementie- rung noch fehlt. Die Aufgabe der jeweiligen Methode ist in der Kommentierung be- schrieben. Ergänzen Sie den zugehörigen Programmcode. Verwenden Sie dazu Metho- den der Klasse String. Die notwendigen Informationen entnehmen Sie der Dokumentati- on der Klasse String.
% Testen Sie Ihren Programmcode ausgiebig. \\




% \newpage
% \input{ha-aufgabe1.tex}
% \input{ha-aufgabe2.tex}
% \input{ha-aufgabex.tex}


% \lstset{language=Ruby, basicstyle=\scriptsize}
% \begin{lstlisting}
% **** ÄÜÖ
% new car ; dsgjhdsjkghsjdg sjkhgjsfdhg sdjkfghjfkdshgjfsd hjkfdsghjfsdkhgsd hjkfdsgjksfdhgjfsdh jkdfsghfjdsgh dfsj
% sdgnjsdgds dsgdsjgsk sdghsjghsjg sgjhfjkhsdjkhgjksdfhgjksfdlg hjkdfshgjfdhsg dsghfdskghfdjskhgsdfjkhfdgfdjhgjklfdhgjkfdhgjfdhgjkfdghdkfjghdfjk
% \end{lstlisting}

% \begin{listingsbox}{myjava}{Test üÜöÖäÄ}
% test
% ÜüÖöÄä
% \end{listingsbox}
% Über
